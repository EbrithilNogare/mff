\documentclass[a4paper]{article}
\usepackage[utf8]{inputenc}   % pro unicode UTF-8
\usepackage[czech]{babel} %jazyk dokumentu
\usepackage{listings}
\usepackage{color}
\usepackage[T1]{fontenc}
\usepackage{amssymb}
\usepackage{hyperref}
\usepackage{listingsutf8}
\usepackage{graphicx}
\usepackage{amsmath}
\usepackage[margin={1cm,2cm}]{geometry}


\newcommand\aug{\fboxsep=-\fboxrule\!\!\!\fbox{\strut}\!\!\!}

\newcommand{\asubsection}[3]{
	\setcounter{section}{#1}
	\setcounter{subsection}{#2}
	\addtocounter{subsection}{-1}
	\subsection{#3}~
}

\newcommand{\dukaz}{
	\subsubsection*{dukaz}~
}

\graphicspath{ {/} }

%%%%%%%%%%%%%%%%%%%%%%%%%%%%%%%%%%%%%%%%%%%%%%%%%%%%%%%%%%%%%

\begin{document}

\section*{Definice}

\asubsection{2}{2}{Matice}
Rálná matice typu $m \times n$ je obdélníkové schema (tabulka)


\asubsection{2}{3}{Vektor}
Reálný n-rozměrný aritmetický sloupcový vektor je matice typu  $m \times 1$


\asubsection{2}{4}{* notace}
i-tý řádek matice $A$ se značí: $A_{i*} = (a_{i1}, a_{i2}, . . . , a_{in})$


\asubsection{2}{5}{Soustava lineárních rovnic}


\asubsection{2}{6}{Matice soustavy}


\asubsection{2}{8}{Elementární řádkové úpravy}


\asubsection{2}{12}{Odstupňovaný tvar matice}


\asubsection{2}{13}{Hodnost matice}


\asubsection{2}{18}{Redukovaný odstupňovaný tvar matice}


\asubsection{3}{1}{Rovnost}


\asubsection{3}{2}{Součet}


\asubsection{3}{3}{Násobek}


\asubsection{3}{7}{Součin}


\asubsection{3}{11}{Transpozice}


\asubsection{3}{14}{Symetrická matice}


\asubsection{3}{23}{Regulární matice}


\asubsection{3}{30}{Inverzní matice}


\asubsection{4}{1}{Grupa}


\asubsection{4}{5}{Podgrupa}


\asubsection{4}{8}{Permutace}


\asubsection{4}{9}{Inverzní permutace}


\asubsection{4}{1}{Skládání permutací}


\asubsection{4}{13}{Znaménko permutace}


\asubsection{4}{22}{Těleso}


\asubsection{4}{35}{Charakteristika tělesa}


\asubsection{5}{1}{Vektorový prostor}


\asubsection{5}{4}{Podprostor}


\asubsection{5}{8}{Lineární obal}


\asubsection{5}{11}{Lineární kombinace}


\asubsection{5}{21}{Lineární nezávislost}


\asubsection{5}{22}{Lineární nezávislost nekonečné množiny}


\asubsection{5}{29}{Báze}


\asubsection{5}{32}{Souřadnice}


\asubsection{5}{42}{Dimenze}


\asubsection{5}{49}{Spojení podprostorů}


\asubsection{5}{55}{Maticové prostory}


\asubsection{6}{1}{Lineární zobrazení}


\asubsection{6}{6}{Obraz a jádro}


\asubsection{6}{14}{Matice lineárního zobrazení}


\asubsection{6}{20}{Matice přechodu}


\asubsection{6}{29}{Isomorfismus}


\asubsection{6}{41}{Prostor lineárních zobrazení}


\asubsection{7}{1}{Afinní podprostor}


\asubsection{7}{7}{Dimenze afinního podprostoru}


\asubsection{7}{10}{Afinní nezávislost}



\newpage
\section*{Věty}

\asubsection{1}{1}{Základní věta algebry}
Každý polynom s komplexními koeficienty má alespoň jeden komplexní kořen.
\dukaz
pres kruznici a jeji zmensovani v rovine komplexnich cisel.
Snizujeme stupen polynomu az na nulu delenim kerenem.


\asubsection{2}{22}{Frobeniova věta}
Soustava $(A | b)$ má (aspoň jedno) řešení právě tehdy, když rank$(A) =$ rank$(A | b)$


\asubsection{3}{28}{o regularni matici}
Buď $A \in R^{m\times n}$.
Pak $RREF(A) = QA$ pro nějakou regulární matici $Q \in R m\times m$
\dukaz
$RREF(A)$ získáme aplikací konečně mnoha elementárních řádkových úprav.
Nechť jdou reprezentovat maticemi $E1, E2, . . . , Ek$.
Pak $RREF(A) = Ek . . . E2E1A = QA$, kde $Q = Ek . . . E2E1$.
Protože matice $E1, E2, . . . , Ek$ jsou regulární, i jejich součin $Q$ je regulární


\asubsection{3}{31}{O existenci inverzní matice}
Buď $A \in R^{n\times n}$.
Je-li $A$ regulární, pak k ní existuje inverzní matice, a je určená jednoznačně.
Naopak, existuje-li k $A$ inverzní, pak $A$ musí být regulární
\dukaz
Existence -  Vytvořme matici $A^{-1}$ tak, aby její sloupce byly vektory $x1, . . . , xn$,
	to jest, $A^{-1} = (x1|x2| . . . |xn)$ \\
Druha rovnost - $A(A^{-1} A - I) = A A^{-1} A - A = IA - A = 0$ \\
Jednoznacnost - $B = BI = B(AA^{-1}) = (BA)A^{-1} = IA^{-1} = A^{-1}$


\asubsection{3}{33}{Jedna rovnost stačí}
Buďte $A, B \in R n\times n$. Je-li $BA = I$,
pak obě matice $A, B$ jsou regulární a navzájem k sobě inverzní,
to jest $B = A^{-1}$ a $A = B^{-1}$
\dukaz
vime ze $I$ je regularni, $B = BI = B(AA^{-1} ) = (BA)A^{-1} = IA^{-1} = A^{-1}$ a obracene


\asubsection{3}{34}{Výpočet inverzní matice}
Buď $A, B \in R^{n\times n}$.
Nechť matice $(A | I_n)$ typu $n\times 2n$ má RREF tvar $(I_n | B)$.
Pak $B = A^{-1}$.
Netvoří-li první část RREF tvaru jednotkovou matici, pak $A$ je singulární
\dukaz
Je-li RREF$(A | I_n) = (I_n | B)$, potom existuje regulární matice $Q$ taková,
že $(I_n | B) = Q(A | I_n)$, neboli po roztržení na dvě části $I_n = QA$ a $B = QI_n$.
První rovnost říká $Q = A^{-1}$ a druhá $B = Q = A^{-1}$.
\\
Netvoří-li první část RREF tvaru jednotkovou matici,
pak RREF$(A) \neq I_n$ a tudíž $A$ není regulární.


\asubsection{3}{37}{Soustava rovnic a inverzní matice}
Buď $A \in R^{n\times n}$ regulární.
Pak řešení soustavy $Ax = b$ je dáno vzorcem $x = A^{-1} b$.
\dukaz
Protože $A$ je regulární, má soustava jediné řešení $x$.
Platí $x = Ix = (A^{-1}A)x = A^{-1} (Ax) = A^{-1} b$


\asubsection{3}{41}{Shermanova–Morrisonova formule}
Buď $A \in R^{n\times n}$ regulární a $b, c \in R^n$.
Pokud $c^T A^{-1} b = -1$, tak $A + bc^T$ je singulární, jinak
$$
(A + bc^T)^{-1} = A^{-1} -\frac{1}{1 + c^TA^{-1}b} A^{-1} bc^TA^{-1}
$$
\dukaz
V případě $c^TA^{-1}b = -1$ máme
$(A + bc^T)A^{-1}b = AA^{-1}b + bc^TA^{-1}b = b(1 + c^TA^{-1}b) = 0$.
Protože $b \neq 0$ a vzhledem k regularitě $A$ je $A^{-1}b \neq 0$,
musí matice $(A + bc^T)$ být singulární


\asubsection{3}{43}{Jednoznačnost RREF}
RREF tvar matice je jednoznačně určen
\dukaz
$A = Q^{-1}_1 A_1 = Q^{-1}_2 A_2$, a tedy $A_1 = Q_1Q^{-1}_2 A_2$ => $A_1 = A_2$


\asubsection{4}{15}{O znaménku složení permutace a transpozice}
Buď $p \in S_n$ a buď $t = (i, j)$ transpozice.
Pak $sgn(p) = - sgn(t \circ p) = - sgn(p \circ t)$

\asubsection{4}{16}{Každou permutaci lze rozložit na složení transpozic}


\asubsection{4}{27}{$Z_n$ je těleso právě tehdy, když $n$ je prvočíslo}
\dukaz
Je-li $n$ složené, pak $n = pq$, kde $1 < p, q < n$. Kdyby $Z_n$ bylo těleso,
pak $pq = 0$ implikuje podle tvrzení 4.25 buď $p = 0$ nebo $q = 0$, ale ani jedno neplatí


\asubsection{4}{33}{O velikosti konečných těles}
Existují konečná tělesa právě o velikostech $p^n$, kde $p$ je prvočíslo a $n \geq 1$ 


\asubsection{4}{38}{Malá Fermatova věta}
Buď $p$ prvočíslo a buď $0 \neq a \in Z_p$. Pak $a^{p-1} = 1$ v tělese $Z_p$


\asubsection{5}{15}{o vektorovem prostoru a obalu}
Buď $V$ vektorový prostor nad $T$, a mějme $v1, . . . , vn \in V$.\\
Pak $span\{v_1, . . . , v_n\} = \{ \sum^n_{i=1} a_iv_i; a_1, . . . , a_n \in T\}$


\asubsection{5}{26}{o vektorove zavislosti}
Buď $V$ vektorový prostor nad $T$, a mějme $v1, . . . , vn \in V$.
Pak vektory $v_1, . . . , v_n$ jsou lineárně závislé právě tehdy,
když existuje $k \in {1, . . . , n}$ takové,
že $v_k = P i \neq k a_iv_i$ pro nějaké $a_1, . . . , a_n \in T$,
to jest $vk \in span\{v_1, . . . , vk-1, v_{k+1}, . . . , vn\}$


\asubsection{5}{31}{o bazi}
Nechť $v_1, . . . , v_n$ je báze prostoru $V$.
Pak pro každý vektor $u \in V$ existují jednoznačně určené
koeficienty $a_1, . . . , a_n \in T$ takové, že $u = \sum^n_{i=1} a_iv_i$

\asubsection{5}{38}{O existenci báze}
Každý vektorový prostor má bázi
\dukaz
Buď $v_1, . . . , v_n$ systém generátorů $V$.
Jsou-li lineárně nezávislé, tak už tvoří bázi.
Jinak podle důsledku 5.27 existuje index $k$ tak,že
$$
span\{v_1, . . . , v_n\} = span\{v_1, . . . , v_{k-1}, v_{k+1}, . . . , v_n\}
$$


\asubsection{5}{40}{Steinitzova věta o výměně}
Buď $V$ vektorový prostor, buď $x_1, . . . , x_m$ lineárně nezávislý
systém ve $V$, a nechť $y_1, . . . , y_n$ je systém generátorů $V$.
Pak platí:\\
1. $m \leq n$ \\
2. existují navzájem různé indexy $k_1, . . . , k_{n-m}$ takové,
že $x_1, . . . , x_m, y_{k_1} , . . . , y_{k_{n-m}}$ tvoří systém generátorů $V$
\dukaz
indukci od $m=0$ predpoklad pro $m-1$ => plati i pro $m$


\asubsection{5}{44}{Vztah počtu prvků systému k dimenzi}
Pro vektorový prostor $V$ platí: \\
1. Nechť $x_1, . . . , x_m$ jsou lineárně nezávislé.
Pak $m \leq dim V$. Pokud $m = dim V$, potom $x_1, . . . , x_m$ je báze. \\
2. Nechť $y_1, . . . , y_n$ jsou generátory $V$ . Pak $n \geq dim V$.
Pokud $n = dim V$, potom $y_1, . . . , y_n$ je báze


\asubsection{5}{45}{Rozšíření lineárně nezávislého systému na bázi}
Každý lineárně nezávislý systém vektorového prostoru $V$ lze rozšířit na bázi $V$


\asubsection{5}{46}{Dimenze podprostoru}
Je-li $W \subseteq V$, pak $dim W \leq dim V$. Pokud navíc $dim W = dim V$, tak $W = V$


\asubsection{5}{50}{Spojení podprostorů}
Buďte $U, V$ podprostory vektorového prostoru $W$. Pak $U + V = span(U \cup V)$


\asubsection{5}{52}{Dimenze spojení a průniku}
Buďte $U, V$ podprostory vektorového prostoru $W$.
Pak platí $dim(U + V ) + dim(U \cap V ) = dim U + dim V$


\asubsection{5}{62}{Maticové prostory a RREF}
Buď $A \in T^{m\times n}$ a buď $A^R$ její $RREF$ tvar
s pivoty na pozicích $(1, p_1), . . . ,(r, p_r)$, kde $r = rank(A)$. Pak: \\
1. nenulové řádky $A^R$, tedy vektory $A^R_{1\star} , . . . , A^R_{r\star}$,
	tvoří bázi $R(A)$ \\
2. sloupce $A_{ \star p_1} , . . . , A_{ \star p_r}$ tvoří bázi $S(A)$ \\
3. $dim R(A) = dim S(A) = r$


\asubsection{5}{63}{Pro každou matici $A \in T^{m\times n}$ platí rank(A) = rank($A^T$)}


\asubsection{5}{66}{O dimenzi jádra a hodnosti matice}
Pro každou matici $A \in T^{m\times n}$ platí dim Ker$(A)+$rank$(A) = n$


\asubsection{6}{10}{Prosté lineární zobrazení}
Buď $f : U \rightarrow V$ lineární zobrazení. Pak následující jsou ekvivalentní: \\
1. $f$ je prosté \\
2. Ker$(f) = \{o\}$ \\
3. obraz libovolné lineárně nezávislé množiny je lineárně nezávislá množina


\asubsection{6}{12}{Lineární zobrazení a jednoznačnost vzhledem k obrazům báze}
Buďte $U, V$ prostory nad $T$ a $x_1, . . . , x_n$ báze $U$.
Pak pro libovolné vektory $y_1, . . . , y_n \in V$ existuje
právě jedno lineární zobrazení takové,
že $f(x_i) = y_i, i = 1, . . . , n$


\asubsection{6}{16}{Maticová reprezentace lineárního zobrazení}
Buď $f : U \rightarrow V$ lineární zobrazení,
$B1 = \{x_1, . . . , x_n\}$ báze prostoru $U$, a
$B2 = \{y_1, . . . , y_m\}$ báze prostoru $V$. \\
Pak pro každé $x \in U$ je $[f(x)]_{B2} = {}_{B2}[f]_{B1} \cdot [x]_{B1}$


\asubsection{6}{18}{Jednoznačnost matice lineárního zobrazení}
Buď $f : U \rightarrow V$ lineární zobrazení,
$B_1$ báze prostoru $U$, a $B_2$ báze prostoru $V$. \\
Pak jediná matice $A$ splňující (6.16) je $A = {}_{B_2}[f]_{B_1}$


\asubsection{6}{24}{Matice složeného lineárního zobrazeni}
Buďte $f : U \rightarrow V$ a $g : V \rightarrow W$ lineární zobrazení,
buď $B_1$ báze $U$, $B_2$ báze $V$ a $B_3$ báze $W$. \\
Pak $ {}_{B_3}[g \circ f]_{B_1} = {}_{B_3}[g]_{B_2} \cdot {}_{B_2}[f]_{B_1}$


\asubsection{6}{35}{Isomorfismus $n$-dimenzionálních prostorů}
Všechny $n$-dimenzionální vektorové prostory nad tělesem T jsou navzájem isomorfní


\asubsection{6}{37}{O dimenzi jádra a obrazu}
Buď $f : U \rightarrow V$ lineární zobrazení,
$U, V$ prostory nad $T$, $B_1$ báze prostoru $U a B_2$ báze prostoru $V$. \\
Označme $A = {}_{B_2}[f]_{B_1}$.
Pak: \\
1. dim Ker$(f) = $ dim Ker$(A)$ \\
2. dim $f(U) = $ dim $S(A) = $ rank$(A)$.


\asubsection{7}{4}{Charakterizace afinního podprostoru}
Buď $V$ vektorový prostor nad tělesem $T$ charakteristiky různé od 2,
a buď $\emptyset \neq M \subseteq V$.
Pak $M$ je afinní, tj. je tvaru $M = U + a$ právě tehdy, když pro každé
$x, y \in M$ a $a \in T$ platí $ax + (1 - a)y \in M$

\asubsection{7}{5}{Množina řešení soustavy rovnic}
Množina řešení soustavy rovnic $A_x = b$ je prázdná nebo afinní.
Je-li neprázdná, můžeme tuto množinu řešení vyjádřit ve
tvaru Ker$(A) + x_0$, kde $x_0$ je jedno libovolné řešení soustav


\end{document}