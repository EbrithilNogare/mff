\documentclass[a4paper]{article}
\usepackage[utf8]{inputenc}
\usepackage{amsmath}
\usepackage{amsfonts}
\usepackage{graphicx}
\usepackage{amssymb}
\usepackage[
top=3cm,
bottom=4cm,
]{geometry}
\setlength{\parindent}{0in}
\usepackage{fancyhdr}

\begin{document}

\pagestyle{fancy}
\rhead{David Nápravník - přezdívka ":)"}

\setcounter{section}{2}
\section{du}
\subsection{sachovnice}
z sachovnice si udelejme jednu dlouhou cestu ktera prochazi kazdym polickem:\\
\begin{tabular}{|l|l|l|l|l|l|}
\hline
\multicolumn{1}{|l}{\#} & \multicolumn{1}{l}{} & \multicolumn{1}{l}{\#} &
\multicolumn{1}{l}{} & \multicolumn{1}{l}{\#} & \\
\cline{2-5}
& \multicolumn{1}{l}{\#} & & \multicolumn{1}{l}{\#} & & \# \\
R & & \# & & \# & \\
& \# & & \# & & \# \\
\# & & \# & & \# & R \\
& \# & & \# & & \# \\
\multicolumn{1}{|l}{\#} & & \multicolumn{1}{l}{\#} & & \multicolumn{1}{l}{\#} &
\\
\hline
\end{tabular}\\
kde \# je cerne policko, cesta je ohranicena a R jsou vymazana policka\\
\\
pak mame ruzne barvy policek na obou koncich cesty a tudiz lze poskladat domino.


\subsection{tok cesta a rez}
\textbf{tvrzeni plati}\\
Mejme graf G kde $f$ je velikost maximalniho toku.\\
Z vety \textit{o maximalnim toku a minimalnim rezu} vime ze,
maximalni velikost toku v siti je rovna minimalni velikosti rezu.\\
Jelikoz cesta $P$ neprochazi na hranach kde celkovy tok je mensi roven nule,
tak musi jit o tok ze zdroje do stoku, ktery se jiz neda zlepsit. 
Potom cesta $P$ bude obsahovat prave jednu hranu z $S$, nebot
neobsahovala-li by hranu, znamenalo by to, ze zde existuje zlepsujici cesta,
ze zdroje do stoku. Nebo obsahovala-li by vice nez jednu hranu, pak
by se do vypoctu velikost minimalniho rezu zapocital dvakrat, ale 
do maximalniho toku by se zapocital pouze jednou. Pote by nesedela rovnost
o min rezu a max toku.



\subsection{kruznice}
pro $n \leq 5$ dostaneme $K_n$ ktery ma $n-1$hranovou a vrcholovou souvislost.\\
pro $n > 5$ pak zustane 4-souvisly jak hranove tak i vrcholove.\\ 
Dukaz sporem, predpokladejme existenci rezu velikosti 3.
To snadno dokazeme pomoci nalezeni maximalniho toku
(tudiz i minimalniho rezu) v grafu $n=5$, pro ostatni n to bude pote platit tez.
Nebot pouze prodluzujeme "obvod" kruznice stejnym patternem. Takovy
min rez bude velikosti 4 -- spor.



\subsection{souvisly graf}
Mejme graf $G(V,E)$ kde plati: $E<30~\&~2E < 5V$.\\
Pri 29 hranach dostaneme 12 vrcholu. A predpokladejme ze kazdy
vrchol je stupne minimalne 5 (tudiz celkem potrebuji $5*12/2=30$ hran).
Pomoci \textit{lemma o holubniku} lehce nahledneme ze
alespon jeden vrchol musi byt stupne nejvyse 4.


\subsection{magicka krychle}
\textbf{ano, plati}\\
Definujme si jednotkovou krychly jakozto krychly, ktera ma na pricne diagonale
jednicky.\\
pak muzeme provadet upravy podobne na maticich, neboli prohozeni poradi dvou
ctvercu(jedne vrstvy) a dostaneme stale krychli sily 1\\
Pak dve krychle sily 1 ktere vzniknou rozlozenim krychle sily dva bodou pouze
nejakou permutaci jednotkove krychle.

\end{document}