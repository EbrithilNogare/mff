\documentclass[a4paper]{article}
\usepackage[utf8]{inputenc}   % pro unicode UTF-8
\usepackage[czech]{babel} %jazyk dokumentu
\usepackage{listings}
\usepackage{color}
\usepackage[T1]{fontenc}
\usepackage{amssymb}
\usepackage{hyperref}
\usepackage{listingsutf8}
\usepackage{graphicx}
\usepackage{amsmath}
\usepackage[margin={1cm,2cm}]{geometry}

\graphicspath{ {/} }

\def\doubleunderline#1{\underline{\underline{#1}}}

%%%%%%%%%%%%%%%%%%%%%%%%%%%%%%%%%%%%%%%%%%%%%%%%%%%%%%%%%%%%%

\begin{document}

\noindent
\textbf{Predmet: Vyrokova a predikatorova logika}\\
\textbf{Ukol: 1.}\\
\textbf{Verze: 1.}\\
\textbf{Autor: David Napravnik}

\section*{a1}
x je minimální prvek $\Leftrightarrow \forall y : y \leq x \implies x=y$


\section*{a2}
x je nejmenší prvek $\Leftrightarrow \forall y : x \leq y$


\section*{b}
x má bezprostředního předchůdce $\Leftrightarrow \exists y : y = {}^-x$


\section*{c}
každé dva prvky mají největšího společného předchůdce\\
$\forall x,y~\exists z : (z = {}^-x \wedge z < y) \vee (z = {}^-y \wedge z < x)$


\end{document}