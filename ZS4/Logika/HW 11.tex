\documentclass[a4paper]{article}
\usepackage[utf8]{inputenc}   % pro unicode UTF-8
\usepackage[czech]{babel} %jazyk dokumentu
\usepackage{listings}
\usepackage{color}
\usepackage[T1]{fontenc}
\usepackage{amssymb}
\usepackage{hyperref}
\usepackage{listingsutf8}
\usepackage{graphicx}
\usepackage{amsmath}
\usepackage[margin={1cm,2cm}]{geometry}

\graphicspath{ {/} }

\def\doubleunderline#1{\underline{\underline{#1}}}

%%%%%%%%%%%%%%%%%%%%%%%%%%%%%%%%%%%%%%%%%%%%%%%%%%%%%%%%%%%%%

\begin{document}

\noindent
\textbf{Predmet: Vyrokova a predikatorova logika}\\
\textbf{Ukol: 11.}\\
\textbf{Verze: 2}\\
\textbf{Autor: David Napravnik}



\section*{axiomy rovnosti}
\renewcommand{\labelenumi}{\roman{enumi}}
\begin{enumerate}
    \item axiom reflexivity: 
        $x=x$
    \item schema axiomu kongruence vzhledem k relacim: $
        x_{1} = y_{1},...,x_{n}=y_{n}
        \rightarrow
        (R(x_{1},...,x_{n})
        \rightarrow
        R(y_{1},...,y_{n}))
        $, kde n je přirozené číslo a R je n-ární relační symbol.
    \item schema axiomu kongruence vzhledem k funkcim: $
        x_{1}=y_{1},...,x_{n}=y_{n}
        \rightarrow
        F(x_{1},...,x_{n})=F(y_{1},...,y_{n})
        $, kde n je přirozené číslo a F je n-ární funkční symbol.
\end{enumerate}


\section*{a)  $T^\star |= x = y \rightarrow y = x$}

\begin{tabular}{ c c c c }
    \multicolumn{4}{c}{$F(\forall x)(\forall y)(x=y\rightarrow y=x)$} \\
    \multicolumn{4}{c}{|} \\
    \multicolumn{4}{c}{$F(c=d\rightarrow d=c)$ nove c, d} \\
    \multicolumn{4}{c}{|} \\
    \multicolumn{4}{c}{$T c=d$}\\
    \multicolumn{4}{c}{$F d=c$}\\
    \multicolumn{4}{c}{|} \\
    \multicolumn{4}{c}{$T(\forall x)(\forall y)(x=x \wedge x=y)\rightarrow(R(x,x)\rightarrow R(y,x))$}\\
    \multicolumn{4}{c}{|} \\
    \multicolumn{4}{c}{$T(c=c \wedge c=d) \rightarrow (R(c,c) \rightarrow R(d,c))$ c=x, d=y}\\ 
    \multicolumn{2}{c}{/} & \multicolumn{2}{c}{$\backslash$} \\
    \multicolumn{2}{c}{$F c=c \wedge c=c$} & \multicolumn{2}{c}{$T R(c,c) \rightarrow R(d,c)$} \\
    / & $\backslash$ & / & $\backslash$ \\
    $F c=c$ & $F c=d$ & $FR(c,c)$ & $TR(d,c)$  \\ 
    | & | & | & | \\
    $T(\forall x)(x=x)$ & $\otimes$ &  $T(\forall x)(x=x)$ & $T d=c$ \\  
    | &  & | & | \\
    $T c=c$ &  & $T c=c$ & $\otimes$ \\
    | &  & | &  \\
    $\otimes$ &  & $\otimes$ &     
\end{tabular}
\\
vsechny vetve jsou sporne, tudiz puvodni tvrzeni plati $\square$



\section*{b)  $T^\star |= (x = y \wedge y = z) \rightarrow x = z$}

\begin{tabular}{ c c c c }
    \multicolumn{4}{c}{$F(\forall x)(\forall y)(\forall z)(x=y \wedge y=z \rightarrow x=z)$} \\
    \multicolumn{4}{c}{|} \\
    \multicolumn{4}{c}{$F(c=d \wedge d=e) \rightarrow c=e$ nove c, d, e} \\
    \multicolumn{4}{c}{|} \\
    \multicolumn{4}{c}{$T c=d \wedge d=e$}\\
    \multicolumn{4}{c}{$F c=e$}\\
    \multicolumn{4}{c}{$T c=d$}\\
    \multicolumn{4}{c}{$T d=e$}\\
    \multicolumn{4}{c}{|} \\
    \multicolumn{4}{c}{$T(\forall x)(\forall y)(\forall z)(x=x \wedge y=z)\rightarrow(R(x,y)\rightarrow R(y,z))$}\\
    \multicolumn{4}{c}{|} \\
    \multicolumn{4}{c}{$T(c=c \wedge d=e) \rightarrow (R(c,d) \rightarrow R(d,e))$ c=x, d=y, e=z}\\ 
    \multicolumn{2}{c}{/} & \multicolumn{2}{c}{$\backslash$} \\
    \multicolumn{2}{c}{$F c=c \wedge d=e$} & \multicolumn{2}{c}{$T R(c,d) \rightarrow R(c,e)$} \\
    / & $\backslash$ & / & $\backslash$ \\
    $F c=c$ & $F d=e$ & $FR(c,d)$ & $TR(c,e)$  \\ 
    | & | & | & | \\
    $T(\forall x)(x=x)$ & $\otimes$ &  $F(c=d)$ & $T c=e$ \\  
    | &  & | & | \\
    $T c=c$ &  & $F c=d$ & $\otimes$ \\
    | &  & | &  \\
    $\otimes$ &  & $\otimes$ &     
\end{tabular}
\\
vsechny vetve jsou sporne, tudiz puvodni tvrzeni plati $\square$


















\end{document}