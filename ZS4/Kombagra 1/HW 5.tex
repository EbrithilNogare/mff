\documentclass[a4paper]{article}
\usepackage[utf8]{inputenc}   % pro unicode UTF-8
\usepackage[czech]{babel} %jazyk dokumentu
\usepackage{listings}
\usepackage{color}
\usepackage[T1]{fontenc}
\usepackage{amssymb}
\usepackage{hyperref}
\usepackage{listingsutf8}
\usepackage{graphicx}
\usepackage{amsmath}
\usepackage{multicol}
\usepackage[margin={1cm,2cm}]{geometry}

\graphicspath{ {/} }

\setcounter{MaxMatrixCols}{40}
\def\doubleunderline#1{\underline{\underline{#1}}}

%%%%%%%%%%%%%%%%%%%%%%%%%%%%%%%%%%%%%%%%%%%%%%%%%%%%%%%%%%%%%

\begin{document}

\noindent
\textbf{Predmet: Kombinatorika a grafy 1}\\
\textbf{Ukol: 5.}\\
\textbf{Verze: 1.}\\
\textbf{Autor: David Napravnik}

\section*{Prvni ukol}
Dokazeme indukci, podle poctu ruznych hodnot\\
Zacneme s balickem o 4 barvach a jednou hodnotou, ten plati trivialne\\
Indukci krok bude pridani jedne hodnoty (ctyr karet ruzne barvy).\\
Pridame balicek 4 karet a pokud karty nechame v tomto balicku, tento balicek bude slouzit pro vybrani dane hodnoty.\\
Pokud by v tomto balicku mela byt karta jine hodnoty, urcime tento balicek pro vybrani one hodnoty a druhy balicekze ktereho kartu bereme
a nastavime na to, abychom z nej vybrali aktualni hodnotu. (neboli prohodime 2 karty a zaroven prohodime co z techto balicku budeme vybirat)\\
toto lze aplikovat i 4krat za sebou (prohozeni celeho balicku) a porad to bude fungovat.
Tudiz porad mame u kazdeho balicku unikatni hodnotu, ktera urcuje jakou kartu z ni budeme vybirat.\\
\\
ukazka: (cislo balicku: 4karty (kterou kartu budeme z balicku vybirat))\\
1: 1111 (1)\\
pridame balicek 2 ...\\
1: 1111 (1)\\
2: 2222 (2)\\
prohodime karty a typy balicku ...\\
1: 1112 (2)\\
2: 2221 (1)











\section*{Druhy ukol}
nejdrive dokazeme pravou implikaci\\
lze vydlazdit => plati pravidlo o poctu der v podtrojuhelnicich\\
dokazeme sporem, predpokladejme, ze neplati pravidlo o poctu der.\\
Pak mame trojuhelnik velikosti n s n+1 dirami.
Jelikoz dira muze byt pouze trojuhelnik smerem nahoru (horni trojuhelnik) a
trojuhelnik velikosti n, ma presne o n hornich trojuhelniku vice nez spodnich.
Tak dostaneme trojuhelnik kde se nerovna pocet hornich a spodnich trojuhelniku.\\
Dalsim pozorovanim je, ze diamant se sklada vzdy z jednoho horniho a jednoho spodniho trojuhleniku.\\
Tudiz jsme dosli ke sporu, nebot takovy trojuhelnik nelze diamanty vyskladat.\\
\\
pote dokazeme levou implikaci\\
plati pravidlo o poctu der v podtrojuhelnicich => lze vydlazdit\\
pokud je pocet der v trojuhleniku roven jeho velikosti, pak jej vydlazdime normalne. (souhlasi pocet hornich a dolnich trojuhelniku)\\
pokud je pocet der mensi nez velikost, pak vydlazdime normalne, ale zbude nam par volnych dolnich trojuhelniku.
To ale nevadi, nebot je vsechny (a ne vic) pouzijeme v nekterem z vetsich trojuhleniku, ktery je nadmnozinou aktualniho trojuhelnika.\\
\\
tudiz plati

\section*{Treti ukol}
\includegraphics[width=300px]{HW5grafy.png}



\section*{Ctvrty ukol}
1) pokud ignorujeme fakt, ze cesty maji stejne pocatecni a koncove vrcholy.
Tak tvrzeni plati\\
nebot $k_v \leq k_e$ tudiz pokud odebereme hrany P1
z grafu, tak graf zustane souvisly a tudiz musi existovat dalsi cesta z x do y.\\
\\
2) Plati\\
odeberme z grafu vrchol z\\
zbyde nam minimalne vrcholove 2-souvisly graf a tudiz minimalne hranove 2-souvisly a tudiz mezi x a y musi existovat minimalne dve disjunktni cesty, coz je definice kruznice
\\
3) Neplati\\
prikladem je graf typu motylek\\










\end{document}