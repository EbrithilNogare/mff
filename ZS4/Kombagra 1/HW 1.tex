\documentclass[a4paper]{article}
\usepackage[utf8]{inputenc}   % pro unicode UTF-8
\usepackage[czech]{babel} %jazyk dokumentu
\usepackage{listings}
\usepackage{color}
\usepackage[T1]{fontenc}
\usepackage{amssymb}
\usepackage{hyperref}
\usepackage{listingsutf8}
\usepackage{graphicx}
\usepackage{amsmath}
\usepackage{multicol}
\usepackage[margin={1cm,2cm}]{geometry}

\graphicspath{ {/} }

\setcounter{MaxMatrixCols}{40}
\def\doubleunderline#1{\underline{\underline{#1}}}

%%%%%%%%%%%%%%%%%%%%%%%%%%%%%%%%%%%%%%%%%%%%%%%%%%%%%%%%%%%%%

\begin{document}

\noindent
\textbf{Predmet: Kombinatorika a grafy 1}\\
\textbf{Ukol: 1.}\\
\textbf{Verze: 1.}\\
\textbf{Autor: David Napravnik}

\section*{Prvni ukol}

$(n!)^3$ \\ >
po roznasobeni porovnavame jeden prvek z $(n!)^3$ s dvemi z $(2n)!$, takze dostaneme $n^3 > 2n*2n-1$ a to plati pro vsechny cleny vyjma $n=1$, ale to je pouze konstanta, tak ji muzeme zanedbat
\\$(2n)!$ \\ >
trivialne
\\$n!$ \\ >
$n! \geq (\frac{n}{e})^n > \log^n(n) \sim \frac{n}{e} > \log(n)$, pro $n>e$
\\$\log^n(n)$ \\ >
$\binom{2n}{n} < 2^{2n} = 4^n < \log^n(n)$
\\$\binom{2n}{n}$ \\ >
trivialne
\\$\binom{2n}{n-1}$ \\ >
podle $2^n < \frac{2^{2n}}{2n+1}\leq \binom{2n}{n}$
\\$2^n$ \\ >
rovnost nastane pro $n=4$ a $n=16$ a pro $n>16$ uz roste rychleji
\\$n^{\sqrt{n}}$ \\ >
$\sqrt{n} > \log(n)$
\\$n^{\log(n)}$ \\ >
$\log(n) > 15$ \dots pro hooodne velka $n$
\\$n^{15}$ \\ >
pocitejme rozklad a pouze nejvyssi mocniny (zbytek prohlasme za konstanty) a mame ($ \binom{2n}{10}\sim n^{10} < n^{15}$)
\\$\binom{2n}{10}$ \\ >
$\binom{2n}{10} > n > \log(n^n)$
\\$\log(n^n)$ \\\\
pouzite definice a vety:\\
$
e(\frac{n}{e})^n \leq n! \leq en(\frac{n}{e})^n  \\
\frac{2^n}{n+1}\leq \binom{n}{\lfloor n/2 \rfloor} \leq 2^n
$




\section*{Druhy ukol}
odpovida Katalanovu cislu, takze\\
\# triangulaci konvexniho $(n+2)-$uhelniku je $C_n = \frac{1}{n+1}\binom{2n}{n}$



\section*{Treti ukol}
$\begin{matrix}
	1&-1&2&-2&3&-3&4&-4& \dots & \sim \\
	\hline
	1&0&2&0&3&0&4&0& \dots & \frac{1}{(1-x^2)^2} \\
	0&-1&0&-2&0&-3&0&-4& \dots & \frac{-x}{(1-x^2)^2}
\end{matrix}$\\
$\frac{1}{(1-x^2)^2} + \frac{-x}{(1-x^2)^2} = \doubleunderline{\frac{1-x}{(1-x^2)^2}}$
\\\\
nedokazu urcit posloupnost podle prvnich prvku, tak se budu ridit zadanym vzorem (stridani $i$ a $2^i$) viz prvni radek\\
$\begin{matrix}
	1&2^1&2&2^2&3&2^3&4&2^4& \dots~i ~ 2^i \\
	1&2&2&4&3&8&4&16& \dots & \sim\\
	\hline
	1 & \cdot & 2 &\cdot & 3 &\cdot & 4 &\cdot & \dots & \frac{1}{(1-x^2)^2}\\
	\cdot & 2 & \cdot & 4 & \cdot & 8 & \cdot & 16 & \dots & \frac{2x}{1-2x^2} 
\end{matrix}$\\
$\doubleunderline{\frac{1}{(1-x^2)^2}+\frac{2x}{1-2x^2}}$





\section*{Ctvrty ukol}
Mejme $a_k = k*2^k$ a $s_n = \sum_{k=0}^n{a_k}$\\
Vytvorujici funkci ziskame z vytvorujici funkce posloupnosti (1,2...), kterou posuneme doprava dosazenim $2x$ a zdvojnasobenim hodnot vynasobenim cislem $2$\\
$a(x) = \frac{2x}{(1-2x)^2}$\\
$s(x)=\frac{2x}{(1-2x)^2(1-x)}$\\
Abychom $s(x)$ dostali v lepsim tvaru pouzijeme rozklad na parcialni zlomky.\\
$s(x)=
\frac{2x}{(1-2x)^2(1-x)}=
\frac{A}{x-1}+\frac{B}{2x-1}+\frac{C}{(2x-1)^2}=
\frac{4}{2 x - 1} + \frac{2}{(2x-1)^2} - \frac{2}{x-1}
$\\
pak prevedeme na sumy\\
$=4\sum^\infty_{n=0}(2x)^n + 2\sum^\infty_{n=0}\binom{n+1}{1}(2x)^n - 2\sum^\infty_{n=0}x^n$\\
$=\sum^\infty_{n=0}((n-1)2^{n+1}+2)x^n$\\
Soucet rady je $\doubleunderline{(n-1)2^{n+1}+2}$







\end{document}