\documentclass[a4paper]{article}
\usepackage[utf8]{inputenc}   % pro unicode UTF-8
\usepackage[czech]{babel} %jazyk dokumentu
\usepackage{listings}
\usepackage{color}
\usepackage[T1]{fontenc}
\usepackage{amssymb}
\usepackage{hyperref}
\usepackage{listingsutf8}
\usepackage{graphicx}
\usepackage{amsmath}
\usepackage{multicol}
\usepackage[margin={1cm,2cm}]{geometry}

\graphicspath{ {/} }

\def\doubleunderline#1{\underline{\underline{#1}}}

%%%%%%%%%%%%%%%%%%%%%%%%%%%%%%%%%%%%%%%%%%%%%%%%%%%%%%%%%%%%%

\begin{document}

\noindent
\textbf{Predmet: Pravděpodobnost a statistika 1}\\
\textbf{Ukol: 6.}\\
\textbf{Verze: 2.}\\
\textbf{Autor: David Napravnik}

\section*{Zadani}
Budte $X,Y,Z\sim Exp(\lambda)$ nezavisle nahodne veliciny

\subsection*{Jake je rozdeleni $X+Y$?}
pomoci konvolunce:\\
$
f_u(u) = \int_{-\infty}^\infty{f_x(x)f_y(u-x)dx} \\
f_u(u) = \int_0^u{\lambda e^{-\lambda x}\cdot\lambda e^{-\lambda(u-x)}~dx} \\
f_u(u) = \int_0^u{\lambda^2 e^{-\lambda(x-(u-x))}~dx} \\
f_u(u) = \int_0^u{\lambda^2 e^{-\lambda u}} \\
f_u(u) = \lambda^2 e^{-\lambda u} \int_0^u{1} \\
f_u(u) = \doubleunderline{\lambda^2 e^{-\lambda u} u} \\
$






\subsection*{Jake je rozdeleni $X+Y+Z$?}
prevedeme na Z+(X+Y)\\
pomoci dvojite konvolunce:\\
$
f_v(v) = \int_{-\infty}^\infty{f_z(z)f_u(v-z)dz} \\
f_v(v) = \int_0^v{\lambda e^{-\lambda z}\cdot\lambda^2 e^{-\lambda (v-z)} x~dz} \\
f_v(v) = \lambda^3 \int_0^v{e^{-\lambda z} e^{-\lambda (v-z)} x~dz} \\
f_v(v) = \lambda^3 \int_0^v{e^{-\lambda (z+v-z)}x~dz} \\
f_v(v) = \lambda^3 \int_0^v{e^{-\lambda v}x~dz} \\
f_v(v) = \lambda^3 e^{-\lambda v} x \int_0^v{1~dz} \\
f_v(v) = \doubleunderline{\lambda^3 e^{-\lambda v} v z} \\
$








\end{document}