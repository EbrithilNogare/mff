\documentclass[a4paper]{article}
\usepackage[czech]{babel} %jazyk dokumentu
\usepackage[margin={1cm,2cm}]{geometry}
\usepackage[T1]{fontenc}
\usepackage[utf8]{inputenc}   % pro unicode UTF-8
\usepackage{amsmath}
\usepackage{amssymb}
\usepackage{color}
\usepackage{graphicx}
\usepackage{hyperref}
\usepackage{listings}
\usepackage{listingsutf8}
\usepackage{multicol}
\usepackage{pgfplots}
\usepackage{tikz}
\usepackage{wrapfig}

\graphicspath{ {/} }

\def\doubleunderline#1{\underline{\underline{#1}}}
\def\cotan{\qopname\relax o{cotan}}
\def\arccotan{\qopname\relax o{arccotan}}

%%%%%%%%%%%%%%%%%%%%%%%%%%%%%%%%%%%%%%%%%%%%%%%%%%%%%%%%%%%%%

\begin{document}

\section*{Vety}
\subsection*{Cauchy-Schwarzova nerovnost}
Pro kazde $x,y \in V$ plati $|\langle x,y \rangle| \leq ||x|| \cdot ||y||$
\\
\textbf{Dukaz}\\
Pro $y=0$ trivialne, pro $y\neq0$:\\
Uvazme realnou funkci $f(t)=\langle x+ty,x+ty \rangle \geq 0$ promene $t\in\mathbb{R}$\\
$f(t)=\langle x,x \rangle + t\langle x,y \rangle + t\langle y,x \rangle + t^2\langle y,y \rangle=
\langle x,x \rangle + 2t \langle x,y \rangle + t^2 \langle y,y \rangle$\\
Ma kladny diskriminant\\
$4\langle x,y \rangle^2 - 4\langle x,x \rangle\langle y,y \rangle \leq 0$\\
z toho dostavame
$\langle x,y \rangle^2\leq\langle x,x \rangle\langle y,y \rangle$


\subsection*{Gram-Schmidtova ortogonalizace (alg. + důkaz spravnost)}
Bud $x_1, ..., x_n \in V$ nezavisle\\
\begin{enumerate}
	\item for $k=1$ to $n$ do
	\item $y_k = x_k - \sum_{j=1}^{k-1}\langle x_k,z_j \rangle z_j$ // kolmice
	\item $z_k = \frac{y_k}{||y_k||}$ // normalizace
	\item end for
\end{enumerate}
Vystup: $z_1, ...,z_n$ ortonormalni baze prostoru $span\{x_1, ...., x_n\}$\\
\textbf{Dukaz}\\
Matematickou indukci podle n\\
Pro $n=1$ je $y_1=x_1\neq 0$ a $z_1 = \frac{x_1}{||x_1||}$ je 
dobre zadefinovane a $span\{x_1\} = span\{z_1\}$\\
indukcni krok $n \gets n-1$\\
...


\subsection*{Ortogonalni projekce}
Bud $V$ vektorovy prostor a $U$ jeho podprostor.
Pak projekci vektoru $x \in V$ rozumime takovy vektor $x_U \in U$,
který splnuje\\
$||x - x_U|| = min_{y\in U} ||x - y||$


\subsection*{radkova linearita determinantu}
Bud $A \in T^{n \times n}$ a $b \in T^n$\\
Pak pro libovolne $i = 1, ..., n$ plati:\\
$det(A + e_ib^T) = det(A) + det(A + e_i(b^T - A_{i*}))$


\subsection*{Determinant soucinu matic / Multiplikativnost determinantu}
Pro kazde $A, B \in T^{n\times n}$ plati $det(AB) = det(A) det(B)$


\subsection*{Laplaceův rozvoj podle radku/sloupce}
Bud $A \in T^{n\times n}$, $n \geq 2$\\
Pak pro kazde $i = 1, ... , n$ plati\\
$det(A) = \sum^n_{j=1}(-1)^{i+j} a_{ij} det(A^{ij})$\\
kde $A^{ij}$ je matice vznikla z $A$ vyskrtnutim i-teho radku a j-teho sloupce


\subsection*{Vlastnosti vlastnich cisel}
Necht $A \in C^{n\times n}$ ma vlastni cisla $\lambda_1, ... , \lambda_n$
a jim odpovidajici vlastni vektory $x_1, ... , x_n$. Pak:
\begin{enumerate}
	\item $A$ je regularni prave tehdy, kdyz 0 neni jeji vlastni cislo
	\item je-li $A$ regularni, pak $A^{-1}$ ma vlastni cisla $\lambda_1^{-1},...,\lambda_n^{-1}$ a vlastni vektory $x_1,...,x_n$
	\item $A^2$ ma vlastni cisla $\lambda_1^2,...,\lambda_n^2$ a vlastni vektory $x_1,...,x_n$
	\item $\alpha A$ ma vlastni cislo $\alpha\lambda_1,...,\alpha\lambda_n$ a vlastni vektory $x_1,...,x_n$
	\item $A+\alpha I_n$ ma vlastni cisla $\lambda_1+\alpha,...,\lambda_n+\alpha$ a vlastni vektory $x_1,...,x_n$
	\item $A^T$ ma vlastni cisla $\lambda_1,...,\lambda_n$, ale vlastni vektory obecne jine
\end{enumerate}

\subsection*{Vlastni cisla podobných matic}
Podobne matice maji stejna vlastnii cisla


\subsection*{Diagonalizovatelnost a baze vlastnich vektorů}
Matice $A \in \mathbb{C}^{n\times n}$
je diagonalizovatelna prave tehdy,
kdyz ma $n$ linearne nezavislych vlastnich vektoru


\subsection*{Vlastni cisla symetrických (Hermitovských) matic}
Vlastni cisla realnych symetrickych (resp. obecneji komplexnich hermitovskych)jsou realna


\subsection*{Spektralni rozklad symetrických matic}
Pro kazdou symetrickou matici $A \in \mathbb{R}^{n\times n}$
existuje ortogonalni $Q \in \mathbb{R}^{n\times n}$ a diagonalni
$V \in \mathbb{R}^{n\times n}$
tak, ze $A = QVQ^T$


\subsection*{Ekvivalentni charakteristiky positivne (semi-)definitnich matic}
Bud $A \in R^{n\times n}$ symetricka.\\
Pak nasledujici podminky jsou ekvivalentni:
\begin{enumerate}
	\item $A$ je positivne (semi)definitni
	\item vlastni cisla $A$ jsou kladna(nezaporna)
	\item existuje matice $U \in R^{m\times n}$ hodnosti $n$ takova, ze $A = U^TU$
\end{enumerate}


\subsection*{Rekuretni test positivni definitnosti}
Symetricka matice 
$A = 
\begin{pmatrix} 
	\alpha     & a^T\\ 
	a & \tilde{A}
\end{pmatrix}
$, kde
$\alpha \in \mathbb{R}, a \in \mathbb{R}^{n-1}$,
$\tilde{A} \in \mathbb{R}^{(n-1)\times(n-1)}$,
je positivne definitni prave tehdy, kdyz
$\alpha > 0$ a $\tilde{A} - \frac{1}{\alpha}aa^T$
je positivne definitni.


\subsection*{Choleskeho rozklad}
Pro kazdou positivne definitni matici $A \in \mathbb{R}^{n\times n}$
existuje jedina dolni trojuhelnikova matice $L \in \mathbb{R}^{n\times n}$
s kladnou diagonalou takova, ze $A = LL^T$


\subsection*{Sylvestrovo kriterium pos. definitnosti}
Symetricka matice $A \in \mathbb{R}^{n\times n}$
je positivne definitni prave tehdy,
kdyz ji Gaussova eliminace prevede do odstupňovaneho tvaru s kladnou diagonalou
za pouziti pouze elementarni upravy pricteni nasobku radku
s pivotem k jinemu radku pod nim.


\subsection*{Skalarni soucin a pos. definitnost}
Operace $\langle x, y\rangle$ je skalarnim soucinem v
$\mathbb{R}^{n}$ prave tehdy, kdyz ma tvar
$\langle x, y\rangle = x^TAy$ pro nějakou positivne definitni matici
$A \in \mathbb{R}^{n\times n}$


\subsection*{Sylvestrův zakon setrvacnosti}
Bud $f(x) = x^TAx$ kvadraticka forma.
Pak existuje baze, vuci niz ma $f$ diagonalni matici s prvky
$1, -1, 0$. Navic, tato matice je az na poradi prvku jednoznacna




























\end{document}