\documentclass[a4paper]{article}
\usepackage[utf8]{inputenc}   % pro unicode UTF-8
\usepackage[czech]{babel} %jazyk dokumentu
\usepackage{listings}
\usepackage{color}
\usepackage{ mathdots }
\usepackage[T1]{fontenc}
\usepackage{amssymb}
\usepackage{hyperref}
\usepackage{listingsutf8}
\usepackage{graphicx}
\usepackage{amsmath}
\usepackage[margin={1cm, 2cm}]{geometry}

\graphicspath{ {/} }

\def\doubleunderline#1{\underline{\underline{#1}}}

%%%%%%%%%%%%%%%%%%%%%%%%%%%%%%%%%%%%%%%%%%%%%%%%%%%%%%%%%%%%%

\begin{document}

\noindent
\textbf{Predmet: Linearni algebra 2}\\
\textbf{Ukol: 2.}\\
\textbf{Verze: 1.}\\
\textbf{Autor: David Napravnik}\\
\textbf{Prezdivka: DN}

\section*{1. zadani}
Urcete charakteristicky polynom, spocitejte vlastni cisla a odpovidajici vlastni vektory

\section*{reseni A}
$(4-\lambda)*(1-\lambda)-(-3 * (-6))$\\
$4-5\lambda+\lambda^2-18$\\
charakteristicky polynom: \doubleunderline{$\lambda^2-5\lambda-14$}\\
\\
vlastni cisla:\\
$\lambda^2-5\lambda-14=0$\\
\doubleunderline{$\lambda_1 = 7$} ; 
\doubleunderline{$\lambda_2 = -2$}\\
\\
vlastni vektory:\\
$
\left[\begin{matrix}
	4-\lambda & -3\\
	-6 & 1-\lambda
	\end{matrix}\right]
$\\
pro $\lambda_1: \left[\begin{matrix}
	1 & 1\\
	0 & 0
	\end{matrix}\right]$
vlastni vektor pro $\lambda_1 = \doubleunderline{[-1,1]}$\\
pro $\lambda_2: \left[\begin{matrix}
	2 & -1\\
	0 & 0
	\end{matrix}\right]$
vlastni vektor pro $\lambda_2 = \doubleunderline{[1,2]}$\\




\section*{reseni B}
$(2-\lambda)^2+1$\\
charakteristicky polynom: \doubleunderline{$\lambda^2-4\lambda+5$}\\
\\
vlastni cisla:\\
$\lambda^2-4\lambda+5=0$\\
\doubleunderline{$\lambda_1 = 2+i$} ; 
\doubleunderline{$\lambda_2 = 2-i$}\\
\\
vlastni vektory:\\
$
\left[\begin{matrix}
	2-\lambda & -1\\
	1 & 2-\lambda
	\end{matrix}\right]
$\\
pro $\lambda_1: \left[\begin{matrix}
	-i & -1\\
	0& 0
	\end{matrix}\right]$
vlastni vektor pro $\lambda_1 = \doubleunderline{[i,1]}$\\
pro $\lambda_2: \left[\begin{matrix}
	i & -1\\
	0 & 0
	\end{matrix}\right]$
vlastni vektor pro $\lambda_2 = \doubleunderline{[-i,1]}$\\






\section*{reseni C}
$-\lambda(1-\lambda)^{2}+\lambda$\\
charakteristicky polynom: \doubleunderline{$2\lambda^2-\lambda^3$}\\
\\
vlastni cisla:\\
$rank~C = 1$, $rank~ker~C = 2$\\
\doubleunderline{$\lambda_1 = \lambda_2 = 0$}\\
$\lambda_1 + \lambda_2 + \lambda_3 = 2$\\
\doubleunderline{$\lambda_3 = 2$}\\
\\
vlastni vektory:\\
$
\left[\begin{matrix}
	1-\lambda & 1 & 0\\
	1 & 1-\lambda & 0\\
	0 & 0 & 0
	\end{matrix}\right]
$\\
pro $\lambda_1: \left[\begin{matrix}
	1 & 1 & 0\\
	0 & 0 & 0\\
	0 & 0 & 0
	\end{matrix}\right]$
vlastni vektor pro $\lambda_1 = \doubleunderline{[-1,1,0]}$\\
pro $\lambda_2: \left[\begin{matrix}
	1 & 1 & 0\\
	0 & 0 & 0\\
	0 & 0 & 0
	\end{matrix}\right]$
vlastni vektor pro $\lambda_2 = \doubleunderline{[0,0,1]}$\\
pro $\lambda_3: \left[\begin{matrix}
	-1 & 1 & 0\\
	0 & 0 & 0\\
	0 & 0 & 0
	\end{matrix}\right]$
vlastni vektor pro $\lambda_3 = \doubleunderline{[1,1,0]}$\\









\section*{2. zadani}
Najdete $\alpha \in \mathbb{R}$ tak, aby $\lambda = 3$
bylo jedno z vlastnich cisel matice M\\


\section*{reseni}
$3+\lambda_2+\lambda_3=trace~M=9$\\
$3*\lambda_2*\lambda_3=det~M=\frac{8\alpha}{3}-8$\\
zvolime nahodne jedno vlastni cislo, druhe dopocitame, treti mame zadane\\
$\lambda_2=2$, $\lambda_3=4$\\
po dosazeni do rovnice tri vlastnich cisel dopocitame pres determinant $\alpha$\\
$3*2*4=\frac{8\alpha}{3}-8$\\
$\doubleunderline{\alpha=6}$\\













\section*{3. zadani}
Najdete nejmensi cislo $\alpha \in \mathbb{R}$ takove, ze matice 
$A+\beta I_n$ je regularni pro vsechna $\beta > \alpha$\\

\section*{reseni}
(pozn. predpokladam ze matici A se mysli matice z prikladu 1,\\
nebot pro obecnou matici, takove $\alpha$ najit nelze,\\
nebot muzeme zvolit ze $A = cI_n; c=-(\alpha+1)$)
\\\\
nejdrive najdeme pro ktera $\beta$ je matice A singularni\\
$\left[\begin{matrix}
	4+\beta & -3\\
	-6 & 1+\beta
\end{matrix}\right]$\\
$\beta^2+5\beta-14=0$\\
$\beta_1 = -7$ ; $\beta_2 = 2$\\
aby byla matice regularni musi platit
$\beta > max(\beta_1, \beta_2) = $ \doubleunderline{$\alpha = 2$}\\













\section*{4. zadani}
Matice $A\in\mathbb{R}^{3\times3}$ ma vlastni cisla
$\lambda_1=-1$, $\lambda_2=2$ a $\lambda_3=5$\\
Urcete stopu a determinant matice $(-A^2+5I_3)^{-1}$ 

\section*{reseni}
vypocitame pres umele vytvorenou matici:\\
$A=\left[\begin{matrix}
	-1 & 0 & 0\\
	0 & 2 & 0\\
	0 & 0 & 5\\
\end{matrix}\right]$\\
$(-A^2+5I_3)^{-1}=
\left[\begin{matrix}
	\frac{1}{4} & 0 & 0\\
	0 & 1 & 0\\
	0 & 0 & \frac{-1}{20}\\
\end{matrix}\right]$\\
$trace~A = \frac{1}{4}+1-\frac{1}{20} = \doubleunderline{\frac{6}{5}}$\\
$det~A = \frac{1}{4}*1*\frac{-1}{20} = \doubleunderline{-\frac{1}{80}}$\\








\end{document}