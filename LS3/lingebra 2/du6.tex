\documentclass[a4paper]{article}
\usepackage[czech]{babel} %jazyk dokumentu
\usepackage[margin={1cm, 2cm}]{geometry}
\usepackage[T1]{fontenc}
\usepackage[utf8]{inputenc}   % pro unicode UTF-8
\usepackage{amsmath}
\usepackage{amssymb}
\usepackage{color}
\usepackage{graphicx}
\usepackage{hyperref}
\usepackage{listings}
\usepackage{listingsutf8}
\usepackage{mathdots}
\usepackage{multicol}
\usepackage{pgfplots}
\usepackage{tikz}
\usepackage{wrapfig}

\graphicspath{ {/} }

\def\doubleunderline#1{\underline{\underline{#1}}}

%%%%%%%%%%%%%%%%%%%%%%%%%%%%%%%%%%%%%%%%%%%%%%%%%%%%%%%%%%%%%

\begin{document}

\noindent
\textbf{Predmet: Linearni algebra 2}\\
\textbf{Ukol: 6.}\\
\textbf{Verze: 1.}\\
\textbf{Autor: David Napravnik}\\
\textbf{Prezdivka: DN}

\section*{zadani}
U nasledujicich matic urcete minimalne 2 zpusoby, zda jsou positivne
semi/definitni\\
$A=
\begin{bmatrix}
	 4 & 1 & -1\\
	 1 & 2 &  1\\
	-1 & 1 &  2\\
\end{bmatrix}
$

\section*{reseni}
\textbf{Pres rekurentni vzorec}\\
$\tilde{A}-\frac{1}{\alpha}aa^T=$\\
$
\begin{bmatrix}
	2 &  1\\
	1 &  2\\
\end{bmatrix}
-\frac{1}{4}[1,-1][1,-1]^T
=
\begin{bmatrix}
	7 &  5\\
	5 &  7\\
\end{bmatrix}
$\\
$
\begin{bmatrix}
	7 &  5\\
	5 &  7\\
\end{bmatrix}
-\frac{1}{7}[5][5]^T
=
\frac{17}{7}
$\\
jelikoz $\frac{17}{7}$ jse kladne tak \doubleunderline{$A$ je positivne definitni}\\
\\
\textbf{Pres Sylvestrovo pravidlo}\\
$\det A = 6$\\
$\det 
\begin{bmatrix}
	4 &  1\\
	1 &  2\\
\end{bmatrix}
= 6$\\
$\det [4] = 4$\\
jelikoz jsou vsechny determinanty kladne je matice $A$ positivne definitni\\






\section*{zadani}
U nasledujicich matic urcete minimalne 2 zpusoby, zda jsou positivne
semi/definitni\\
$B=
\begin{bmatrix}
	1 & 2 & 3\\
	2 & 2 & 4\\
	3 & 4 & 2\\
\end{bmatrix}
$\\
\\
Vime, ze jedna z nich positivne definitni neni. Zmente jeden jeji prvek tak, aby positivne
definitni byla
 
\section*{reseni}
\textbf{Pres rekurentni vzorec}\\
$\tilde{A}-\frac{1}{\alpha}aa^T=$\\
$
\begin{bmatrix}
	2 &  4\\
	4 &  2\\
\end{bmatrix}
-\frac{1}{1}[2,3][2,3]^T
=
\begin{bmatrix}
	-2 & -2 \\
	-2 & -7 \\
\end{bmatrix}
$\\
$
[-7]-\frac{1}{-2}[-2][-2]^T
=
-5
$\\
$-5$ neni positivne definitni tudiz \doubleunderline{$B$ neni positivne definitni}\\
\\
\textbf{Pres Sylvestrovo pravidlo}\\
$\det B = 10$\\
$\det 
\begin{bmatrix}
	1 &  2\\
	2 &  2\\
\end{bmatrix}
= -2$\\
$\det [1] = 1$\\
jelikoz nejsou vsechny determinanty kladne, neni matice $B$ positivne definitni\\
\\
\textbf{Predelani matice aby byla positivne definitni}\\
matici nelze upravit na positivne definitni zmenou pouze jednoho prvku\\
Dukaz:\\
Puvodni matice neprosla pres Sylvestrovo pravidlo na velikosti $2\times2$\\
tudiz chceme najit v tomto useku prvek a zmenit jej
aby determinanty vsech hlavnich vedoucich podmatic matic
byly kladne\\
prvky muzeme menit pouze na diagonale, abychom neporusily symetrii\\
tudiz muzeme zmenit prvky $b_{11}$ a $b_{22}$\\
\\
prvek $b_{11}$ by musel splnovat:\\
$b_{11}<\frac{11}{6} ~\&~ b_{11}>2 ~\&~ b_{11}>0$, coz nelze, tudiz to neni hledany prvek\\
\\
prvek $b_{22}$ by musel splnovat:\\
$b_{22}<\frac{24}{7} ~\&~ b_{22}>4$, coz nelze, tudiz to neni hledany prvek\\
\\
zadny dalsi prvek nemuzeme zmenit, tudiz z teto matice
positivne definitni vyrobit nelze
 









\section*{zadani}
Urcete minimalne 2 zpusoby, zda je nasledujici matice radu n
positivne definitni

\section*{reseni}
pro \textbf{Sylvestrovo pravidlo} zjistime determinanty:\\
$\det C_n = n+1$ , kde $n$ je rad matice\\
tudiz pro kazdy rad matice $C$ je determinant kladny,\\
tudiz matice $C$ je positivne definitni\\
\\
Jako druhy zpusob pouzijeme \textbf{rekurentni vzorec}:\\
posledni cast rekurentniho vzorce bude:\\
$2-\frac{n+1}{n}$, kde $n$ je rad matice.\\
A jelikoz $2-\frac{n+1}{n}$ je vzdy kladne, tak cela matice $C$ je positivne definitni.









\section*{zadani}
Urcete vsechny matice $D \in R^{n\times	n}$,
takove, ze $D$ i $-D$ jsou positivne semidefinitni

\section*{reseni}
Mejme vetu 11.8 (Charakterizace positivni semidefinitnosti),
ta nam rika ze nasledujici podminky jsou ekvivalentni:\\
\begin{itemize}
	\item $A$ je positivne semidefinitni
	\item vlastni cisla $A$ jsou nezaporna
\end{itemize}
dale mejme tvrzeni: $\lambda_{Dx} = -\lambda_{-Dx}$ | $\lambda_{-Dx}$ mysleno jako $x$te vlastni cislo matice $-D$\\
tudiz matice \doubleunderline{$D$ musi byt nulova}






















\section*{zadani}
Bud $E$ positivne semidefinitni a $e_{ii} = 0$ pro jiste $i$.
Ukazte, ze $i$-ty radek a $i$-ty sloupec matice $E$ jsou nulove.

\section*{reseni}
krasne je to videt jiz na male matici $2\times2$\\
$A=
\begin{bmatrix}
	1 & 1 \\
	1 & 0 \\
\end{bmatrix}
$\\
$[a,b] 
\begin{bmatrix}
	1 & 1 \\
	1 & 0 \\
\end{bmatrix}
\begin{bmatrix}
	a \\
	b \\
\end{bmatrix}
=
a^2+2ab
$\\
tim ze mame na diagonale nulu, nikde se nam neobjevi $b^2$,
ale zaroven mame ve sloupcich a radcich nenulove hodnoty,
tudiz se nam tam objevi $b$.\\
A za $b$ pak lze dosadit cislo podle $a$ takove,
ze vysledek bude zaporny. 




























\end{document}