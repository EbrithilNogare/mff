\documentclass[a4paper]{article}
\usepackage[utf8]{inputenc}   % pro unicode UTF-8
\usepackage[czech]{babel} %jazyk dokumentu
\usepackage{listings}
\usepackage{color}
\usepackage[T1]{fontenc}
\usepackage{amssymb}
\usepackage{hyperref}
\usepackage{listingsutf8}
\usepackage{graphicx}
\usepackage{amsmath}
\usepackage[margin={1cm,2cm}]{geometry}
\usepackage{pgfplots} % render grafu


\newcommand\aug{\fboxsep=-\fboxrule\!\!\!\fbox{\strut}\!\!\!}

\newcommand{\definice}[3]{
	\setcounter{section}{#1}
	\setcounter{subsection}{#2}
	\addtocounter{subsection}{-1}
	\subsection{#3}~
}

\newcommand{\veta}[3]{
	\setcounter{section}{#1}
	\setcounter{subsection}{#2}
	\addtocounter{subsection}{-1}
	\subsection{#3}~
}

\newcommand{\dukaz}{
	\subsubsection*{dukaz}~
}

\graphicspath{ {/} }

%%%%%%%%%%%%%%%%%%%%%%%%%%%%%%%%%%%%%%%%%%%%%%%%%%%%%%%%%%%%%

\begin{document}

\textbf{skalarni soucin}\\
\begin{itemize}
    \item $||v|| \geq 0$ a $0$ nastane pouze pro $v=0$
    \item $||\alpha v|| = |\alpha| ||v||$
    \item $||u+v|| \leq ||u|| + ||v||$ 
\end{itemize}

\textbf{priklady norem:}\\
na jednotkove kruznici: (manhatonova norma)\\
$1=\sqrt{x^2+(x-y)^2+y^2}$\\
nam vykresli elipsu\\
\\
cebisevova norma nam vykresli "ctverec" kde se s rostoui odmocninou kulati rohy\\
\\
\textbf{tvrzeni:}\\
pro normy ind. skalarnich soucinem plati:\\
$ ||x-y||^2 + ||x+y||^2 = 2||x||^2 + 2||y||^2 $\\
\\
\textbf{$u$ a $v$ jsou kolme prave kdyz:}
$\langle u | v \rangle = 0$\\
\\
\section*{28.02.}
$$||x|| = \sqrt{\langle x,x \rangle}$$
$$z_i' = \frac{z_i}{||z_i||}$$
$$x_2 = \langle x_2 | z_1 \rangle z_1 + \langle{ x_2 | z_2 \rangle z_2};~y_2=x_2 - \langle y_2|z_1\rangle z_1$$
\\
$x_1 = (2,0,1,2)^T$\\
$x_2 = (4,3,2,4)^T$\\
$x_3 = (6,-5,3,6)^T$\\
$x_4 = (6,-5,3,6)^T$\\
... znormalizujeme\\
$z_1 = (\frac{2}{3}, 0, \frac{1}{3}, \frac{2}{3})$\\
$z_2 = (0,1,0,0)$\\
$y_3 = (0,0,0,0)$\\
$z_4 = (\frac{1}{3}, 0, \frac{2}{3}, \frac{-2}{3})$\\
\\



\section*{6.3.}
test na ortogonalni vektory v $R^3$\\
\\
$x_1 = (1,2,3,4)^T$ \dots $||x_1|| = \sqrt{30}$\\
$x_2 = (2,4,2,1)^T$ \dots $||x_2|| = \sqrt{25} = 5$\\
$x_3 = (-1,-2,-2,-1)^T$ \dots $||x_3|| = \sqrt{10}$\\
\\
Gram–Schmidt:\\
$proj_u(v) = \frac{\langle v,u \rangle}{\langle u,u \rangle} u$\\
\\
$u_1 = x_1 = (1,2,3,4)^T$\\
$u_2 = x_2 - proj_{u_1}(v_2)$\\
$u_3 = x_3 - proj_{u_1}(v_3) - proj_{u_2}(v_3)$\\
\\
normalizace:\\
$z_i=\frac{y_i}{||y_i||}$\\
\\
$A = \left(
	\begin{matrix} 
		1 & 1 & 1 & 1 \\
		4 & 1 & 4 & 1 \\
		1 & 2 & 3 & 4  \end{matrix}
	\right) $\\
$x=(2,2,1,5)^T$\\
Urcete projekci $x$ do $R(A)$\\
\\
Fourierovy koeficienty:\\
$x = \sum^{n}_{i=1}{\langle x, z_i \rangle }z_i$\\
\\

\section*{6.3.}
Spoctete vzdalenost bodu $A=(5,5,3,3)^T$\\
od roviny prochazejici body:\\
\begin{itemize}
	\item $B = (8,-1,1,-2)^T$
	\item $C = (4,-2,2,-1)^T$
	\item $D = (0,0,0,0)^T$
\end{itemize}









\end{document}