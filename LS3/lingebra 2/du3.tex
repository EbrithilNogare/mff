\documentclass[a4paper]{article}
\usepackage[utf8]{inputenc}   % pro unicode UTF-8
\usepackage[czech]{babel} %jazyk dokumentu
\usepackage{listings}
\usepackage{color}
\usepackage{ mathdots }
\usepackage[T1]{fontenc}
\usepackage{amssymb}
\usepackage{hyperref}
\usepackage{listingsutf8}
\usepackage{graphicx}
\usepackage{amsmath}
\usepackage[margin={1cm, 2cm}]{geometry}

\graphicspath{ {/} }

\def\doubleunderline#1{\underline{\underline{#1}}}

%%%%%%%%%%%%%%%%%%%%%%%%%%%%%%%%%%%%%%%%%%%%%%%%%%%%%%%%%%%%%

\begin{document}

\noindent
\textbf{Predmet: Linearni algebra 2}\\
\textbf{Ukol: 2.}\\
\textbf{Verze: 1.}\\
\textbf{Autor: David Napravnik}\\
\textbf{Prezdivka: DN}

\section*{1. zadani}
Prevedte nasledujici matice do tvaru $SDS^{-1}$,
kde $D$ je diagonalni a $S$ je regularni.

\section*{reseni A}
$\left[\begin{matrix}
	0-\lambda & -3 & -3\\
	-4 & -7-\lambda & -7\\
	6 & 12 & 12-\lambda\\
\end{matrix}\right]$\\
Vlastni cisla:\\
$p_A(\lambda) = 
+(0-\lambda)*(-7-\lambda)*(12-\lambda)
+(-3)*(-7)*(6)
+(-3)*(-4)*(12)
-(-3)*(-7-\lambda)*(6)
-(-7)*(12)*(0-\lambda)
-(-3)*(-4)*(12-\lambda)
= -\lambda^3+5\lambda^2-6\lambda
$\\
$
\lambda_1 = 0\\
\lambda_2 = 2\\
\lambda_3 = 3\\
$\\
$D=\left[\begin{matrix}
	0 & 0 & 0\\
	0 & 2 & 0\\
	0 & 0 & 3\\
\end{matrix}\right]$
\\\\
Vlastni vektory:\\
pro $\lambda_1$: 
$\left[\begin{matrix}
	0 & -3 & -3\\
	-4 & -7 & -7\\
	6 & 12 & 12\\
\end{matrix}\right] 
=> [0,-1,1]$
\\
pro $\lambda_2$: 
$\left[\begin{matrix}
	-2 & -3 & -3\\
	-4 & -9 & -7\\
	6 & 12 & 10\\
\end{matrix}\right]
=> [-3,-1,3]$
\\
pro $\lambda_3$: 
$\left[\begin{matrix}
	-3 & -3 & -3\\
	-4 & -10 & -7\\
	6 & 12 & 9\\
\end{matrix}\right]
=> [1,1,-2]$\\
$S=\left[\begin{matrix}
	0 & -3 & 1\\
	-1 & -1 & 1\\
	1 & 3 & -2\\
\end{matrix}\right]$
\\\\
\doubleunderline{
$
\left[\begin{matrix}
	0 & -3 & 1\\
	-1 & -1 & 1\\
	1 & 3 & -2\\
\end{matrix}\right]
\left[\begin{matrix}
	0 & 0 & 0\\
	0 & 2 & 0\\
	0 & 0 & 3\\
\end{matrix}\right]
\left[\begin{matrix}
	0 & -3 & 1\\
	-1 & -1 & 1\\
	1 & 3 & -2\\
\end{matrix}\right]^{-1}
=
\left[\begin{matrix}
	0 & -3 & -3\\
	-4 & -7 & -7\\
	6 & 12 & 12\\
\end{matrix}\right]
$
}



\section*{reseni B}
$
\left[\begin{matrix}
	-1-\lambda & 1\\
	-1 & -1-\lambda\\
\end{matrix}\right]
$\\
Vlastni cisla:
$p_A(\lambda) = (-1-\lambda)^2+1=\lambda^2+2\lambda+2$\\
$\lambda_1 = -1+i$\\
$\lambda_2 = -1-i$\\
Vlastni vektory:\\
pro $\lambda_1$: 
$\left[\begin{matrix}
	-i & 1\\
	-1 & -i\\
\end{matrix}\right]
=> [-i,1]$
\\
pro $\lambda_2$: 
$\left[\begin{matrix}
	i & 1\\
	-1 & i\\
\end{matrix}\right]
=> [i,1]$
\\




\section*{2. zadani}
Rozhodnete o platnosti nasledujicich implikacich


\section*{reseni}
1.) Plati\\
Mejme $\lambda_{Ax}$ jsou vlastni cisla matice $A$,\\
a $\lambda_{Bx}$ jsou vlastni cisla matice $A^2$,\\
pak pro kazde z nich plati: $\lambda_{Bx}=(\lambda_{Ax})^2$
\\\\
2.) Neplati\\
Mejme $\lambda_{Ax}$ jsou vlastni cisla matice $A$,\\
a $\lambda_{Bx}$ jsou vlastni cisla matice $A^2$,\\
pak pro kazde z nich musi platit: $\lambda_{Bx}=(\lambda_{Ax})^2$, ale\\
to neplati pro $\lambda_{Ax} < 0$














\section*{3. zadani}
Bud $A\in \mathbb{R}^{n\times n}$ diagonalizovatelna. Ukazte $A \sim A^T$

\section*{reseni}
Jelikoz $(A - \lambda I)^T = (A^T - \lambda I)$\\
Pak plati: $\det(A^T - \lambda I) = \det((A - \lambda I)^T)=\det(A-\lambda I)$ \\
Tudiz matice $A$ a matice $A^T$ maji stejne vlastni cisla a tudiz jsou podobne










\section*{4. zadani}
Budte $A, B \in\mathbb{R}^{n\times n}$ podobne.
Ukazte, ze maticova soustava $AX - XB = 0$
ma reseni $X\in\mathbb{R}^{n\times n}$

\section*{reseni}
$A=SBS^{-1}$ | vzorecek podobnosti\\
$SBS^{-1}X = XB$ | dokazovana rovnice\\
$SBS^{-1}XX^{-1} = XBX^{-1}$\\
$SBS^{-1} = XBX^{-1}$ | \\
pak vidime ze $X=S$










\end{document}