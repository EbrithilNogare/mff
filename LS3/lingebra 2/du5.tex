\documentclass[a4paper]{article}
\usepackage[czech]{babel} %jazyk dokumentu
\usepackage[margin={1cm, 2cm}]{geometry}
\usepackage[T1]{fontenc}
\usepackage[utf8]{inputenc}   % pro unicode UTF-8
\usepackage{amsmath}
\usepackage{amssymb}
\usepackage{color}
\usepackage{graphicx}
\usepackage{hyperref}
\usepackage{listings}
\usepackage{listingsutf8}
\usepackage{mathdots}
\usepackage{multicol}
\usepackage{pgfplots}
\usepackage{tikz}
\usepackage{wrapfig}

\graphicspath{ {/} }

\def\doubleunderline#1{\underline{\underline{#1}}}

%%%%%%%%%%%%%%%%%%%%%%%%%%%%%%%%%%%%%%%%%%%%%%%%%%%%%%%%%%%%%

\begin{document}

\noindent
\textbf{Predmet: Linearni algebra 2}\\
\textbf{Ukol: 5.}\\
\textbf{Verze: 1.}\\
\textbf{Autor: David Napravnik}\\
\textbf{Prezdivka: DN}

\section*{zadani}
Jaka je pravdepodobnost, ze pozitri bude slunecno, pokud dnes bylo destivo?

\section*{reseni}
matice pocasi: $
S=
\begin{bmatrix}
	.8 & .2\\
	.6 & .4\\
\end{bmatrix}
$\\
dnesek $=S^0$\\
zitrek $=S^1$\\
pozitri $=S^2$\\
$
\begin{bmatrix}
	.8 & .2\\
	.6 & .4\\
\end{bmatrix}^2
=
\begin{bmatrix}
	.76 & .24\\
	.72 & .28\\
\end{bmatrix}
$\\
dnes bylo destivo, tudiz nas zajima druhy radek\\
pravdepodobnosti odpovida vektoru $[.72, .28]$\\
\doubleunderline{Pozitri bude na $72\%$ slunecno}






\section*{zadani}
Jake je limitni rozlozeni pravdepodobnosti za delsi casovy horizont?

\section*{reseni}
matice pocasi: $
S=
\begin{bmatrix}
	.8 & .2\\
	.6 & .4\\
\end{bmatrix}
$\\
rozklad matice: $M=SJS^{-1}$\\
$
S=
\begin{bmatrix}
	-\frac{1}{3} & 1\\
	1 & 1\\
\end{bmatrix}\\
S^{-1}=
\begin{bmatrix}
	-.75 & .75\\
	.75 & .25\\
\end{bmatrix}\\
J=
\begin{bmatrix}
	.2 & 0\\
	0 & 1\\
\end{bmatrix}\\
\lim_{n\rightarrow\infty}J^n=
\begin{bmatrix}
	0 & 0\\
	0 & 1\\
\end{bmatrix}\\
$\\
\\
$ \lim_{n\rightarrow\infty}M^n=SJ^nS^{-1}$\\
$ \lim_{n\rightarrow\infty}M^n=
\begin{bmatrix}
	-\frac{1}{3} & 1\\
	1 & 1\\
\end{bmatrix}
\begin{bmatrix}
	0 & 0\\
	0 & 1\\
\end{bmatrix}
\begin{bmatrix}
	-.75 & .75\\
	.75 & .25\\
\end{bmatrix}
$\\
$ \lim_{n\rightarrow\infty}M^n=
\begin{bmatrix}
	.75 & .25\\
	.75 & .25\\
\end{bmatrix}
$\\
\doubleunderline{Limitni rozlozeni odpovida vektoru $M^{\infty}x_0 = [.75,.25]$}
(75\% slunecno, 25\% destivo)

















\section*{zadani}
Charakterizujte (vcetne konstrukce matice prechodu)
populaci brouku v 1., 2., 3. a 6. roce za predpokladu,
ze vychozi populace obsahovala 3000 brouku (vsichni brouci
jsou stejne stari a prave se narodili)

\section*{reseni}
matice prechodu: $
P=
\begin{bmatrix}
	0 & \frac{1}{2} & 0\\
	0 & 0 & \frac{1}{3}\\
	6 & 0 & 0 \\
\end{bmatrix}
$\\
pocatecni stav: $B=[3000, 0, 0]$\\
stav po $n$ letech: $BP^{n\%3}$ | kazte tri roky se cyklus opakuje\\
\\
po 1 roce:  $BP^{1} = [0,1500,0]$\\
po 2 roce:  $BP^{2} = [0,0,500]$\\
po 3 roce:  $BP^{0} = [3000,0,0]$\\
po 6 roce:  $BP^{0} = [3000,0,0]$\\









\section*{zadani}
Jak se populace vyviji v case jdoucim do nekonecna?
Zavisi tento vyvoj na velikosti vychozi populace?

\section*{reseni}
jak bylo zmineno v predchozi uloze,
pocty brouku se cykli po 3 letech.\\
V nekonecnu, tedy brouci \textbf{nevyhynou, ani se nepremnozi}
(bude jich konstantne mnoho).\\
Na velikosti populace \textbf{nezalezi}
(dokud jich v kazce fazi cycklu bude $>1$)
\\\\
pozn. musim pochvalit ulohu, velmi se mi libila









\section*{zadani}
Urcete Gerschgorinovy disky pro matici $A$
a rozhodnete, zda ma matice $A$
aspon jedno realne zaporne vlastni cislo.\\
$A=
\begin{bmatrix}
	4 & 0 & 2\\
	-2 & 8 & 2\\
	0 & 2 & -4 \\
\end{bmatrix}
$


\section*{reseni}

\begin{wrapfigure}{r}{0.5\linewidth}
\begin{tikzpicture}
    \begin{axis}[		
		axis y line=middle,
		axis x line=middle,
        xmin=-7, xmax=13, ymin=-5, ymax=5,
        axis equal,
        ]
        \draw[fill=lightgray] (axis cs: 8, 0) circle [radius=40];
        \draw[fill=lightgray] (axis cs: 4, 0) circle [radius=20];
        \draw[fill=lightgray] (axis cs: -4, 0) circle [radius=20];
	\end{axis}
    \begin{axis}[		
		axis y line=middle,
		axis x line=middle,
        xmin=-7, xmax=13, ymin=-5, ymax=5,
        axis equal,
        ]
    \end{axis}
\end{tikzpicture}
\end{wrapfigure}

stredy:\\
\indent$c_1=a_{11}=4$\\
\indent$c_2=a_{22}=8$\\
\indent$c_3=a_{33}=-4$\\
polomery:\\
\indent$r_1=|a_{12}|+|a_{13}|=|0|+|2|=2$\\
\indent$r_2=|a_{21}|+|a_{23}|=|-2|+|2|=4$\\
\indent$r_3=|a_{31}|+|a_{32}|=|0|+|2|=2$\\
Matice $A$ \textbf{ma prave jedno vlastni zaporne cislo} $\lambda_1\in\mathbb{C}$,
o tom zda je realne, ale pomoci Gerschgorinovych disku zjistit nelze


















\end{document}