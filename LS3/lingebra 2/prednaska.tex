\documentclass[a4paper]{article}
\usepackage[utf8]{inputenc}   % pro unicode UTF-8
\usepackage[czech]{babel} %jazyk dokumentu
\usepackage{listings}
\usepackage{color}
\usepackage[T1]{fontenc}
\usepackage{amssymb}
\usepackage{hyperref}
\usepackage{listingsutf8}
\usepackage{graphicx}
\usepackage{amsmath}
\usepackage[margin={1cm,2cm}]{geometry}


\newcommand\aug{\fboxsep=-\fboxrule\!\!\!\fbox{\strut}\!\!\!}

\newcommand{\definice}[3]{
	\setcounter{section}{#1}
	\setcounter{subsection}{#2}
	\addtocounter{subsection}{-1}
	\subsection{#3}~
}

\newcommand{\veta}[3]{
	\setcounter{section}{#1}
	\setcounter{subsection}{#2}
	\addtocounter{subsection}{-1}
	\subsection{#3}~
}

\newcommand{\dukaz}{
	\subsubsection*{dukaz}~
}

\graphicspath{ {/} }

%%%%%%%%%%%%%%%%%%%%%%%%%%%%%%%%%%%%%%%%%%%%%%%%%%%%%%%%%%%%%

\begin{document}


\definice{8}{2}{skalarni soucin nad $\mathbb{R}$}
Bud $V$ vektorovy prostor nad $\mathbb{R}$.\\
Pak skalarni soucin je zobrazeni $\langle\cdot , \cdot\rangle:V^2 \rightarrow \mathbb{R}$,
splnujici $\forall x, y, z \in V, |forall \alpha \in \mathbb{R}$\\
\begin{itemize}
    \item $\langle x, y\rangle \geq 0$ a rovnost nastane pouze pro $x=0$
    \item $\langle x+y,z\rangle = \langle x,z\rangle + \langle y,z\rangle$
    \item $\langle\alpha x, y\rangle = \alpha\langle x,y\rangle$
    \item $\langle x,y\rangle = \langle y,x\rangle$
\end{itemize}


\definice{8}{3}{skalarni soucin nad $\mathbb{C}$}
to same jako 8.2 se zmenou:\\
\begin{itemize}
    \item $\langle x,y\rangle = \overline{\langle y,x\rangle}$
\end{itemize}


\definice{8}{8}{Norma indukovana skalarnim soucinem}
Norma indukovana skalarnim soucinem je definovano pro
$x\in V$ jako $ ||x|| = \sqrt{\langle x,x\rangle} $

\definice{8}{9}{Kolmost}
vektory $x,y \in V$ jsou kolme pokud $\langle x, y \rangle = 0$


\veta{8}{11}{Pythagorova}
Pokud $x,y \in V$ jsou kolme, tak $||x+y||^2 = ||x||^2+||y||^2$

\dukaz
$||x+y||^2=\langle x+y, x+y\rangle=\langle x,x\rangle+\langle x,y\rangle+\langle y,z\rangle+\langle y,y\rangle=$\\
dle definice 8.2 item 2.\\
$\langle x+y, x+y\rangle=\langle x,x+y\rangle+\langle y,x+y\rangle$\\
dle predpokladu kolmosti $\langle x,y\rangle = \langle y,x\rangle = 0$\\
$\langle x,x\rangle = ||x||^2$ a $\langle y,y\rangle = ||y||^2$


\veta{8}{13}{Cauchyho-Schwarzova nerovnost}
Pro kazde $x,y \in V$ plati $|\langle x,y\rangle| \leq ||x||\cdot||y||$


\definice{8}{15}{Norma}
Bud $V$ vektorovy prostor nad $\mathbb{R}$ resp. $\mathbb{C}$\\
Pak norma je zobrazeni $||/cdot|| : V \rightarrow \mathbb{R}$,
splnujici $\forall x, y \in V a \forall \alpha \in \mathbb{R}$\\
\begin{itemize}
    \item $||x|| \geq 0$ a rovnost nastane pouze pro $x=0$
    \item $||\alpha x || = |\alpha|\cdot ||x||$
    \item $||x+y|| \leq |||x|| + ||y||$
\end{itemize}








\end{document}