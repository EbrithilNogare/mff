\documentclass[a4paper]{article}
\usepackage[utf8]{inputenc}   % pro unicode UTF-8
\usepackage[czech]{babel} %jazyk dokumentu
\usepackage{listings}
\usepackage{color}
\usepackage[T1]{fontenc}
\usepackage{amssymb}
\usepackage{hyperref}
\usepackage{listingsutf8}
\usepackage{graphicx}
\usepackage{amsmath}
\usepackage[margin={1cm,2cm}]{geometry}

\graphicspath{ {/} }

\def\doubleunderline#1{\underline{\underline{#1}}}

%%%%%%%%%%%%%%%%%%%%%%%%%%%%%%%%%%%%%%%%%%%%%%%%%%%%%%%%%%%%%

\begin{document}

\noindent
\textbf{Predmet: Mataliza 1}\\
\textbf{Ukol: 4.}\\
\textbf{Verze: 1.}\\
\textbf{Autor: David Napravnik}\\
\textbf{Prezdivka: DN}

\section*{zadani}
Spoctete $lim_{n \rightarrow \infty} \sqrt[n]{n^2 + 1}$


\section*{reseni}
Zacneme, ze si rovnici upravime na tvar 
$lim_{n \rightarrow \infty} \sqrt[n]{n^2 + 1^2 + 2n}$\\\\
rovnici pote zjednodusime na
$lim_{n \rightarrow \infty} (n + 1)^{2/n}$
coz muzeme, nebot odmocnina nemohla byt zaporna\\\\
Jelikoz polynom roste radove rychleji nez obycejne $n$,
pak muzeme 1. zahodit dvojku ve zlomku a ponechat pouze
$\frac{1}{n}$ a 2. prohlasit, ze cokoliv na nultou je jedna.\\
nebo vyuzit lemma $lim_{n \rightarrow \infty} \sqrt[n]{n}=1$\\\\
Pote upravime rovnici na tvar
$lim_{n \rightarrow \infty} \sqrt[n]{n^2 + 1^2 - 2n}$\\\\
rovnici pote zjednodusime na
$lim_{n \rightarrow \infty} (n - 1)^{2/n}$
a postupujeme jako u rovnice vyse.\\\\
Dale podle vety o Policajtech:\\
$lim_{n \rightarrow \infty} \sqrt[n]{n^2 + 1^2 - 2n} \leq
lim_{n \rightarrow \infty} \sqrt[n]{n^2 + 1^2} \leq
lim_{n \rightarrow \infty} \sqrt[n]{n^2 + 1^2 + 2n}$\\
$\doubleunderline{lim_{n \rightarrow \infty} \sqrt[n]{n^2 + 1^2} = 1}$





\section*{zadani}
Spoctete limitu posloupnosti zadane 
$a_1 = 1, a_{n+1} = \frac{a^2_n}{4} + 1$


\section*{reseni}
Nejdrive zkusme najit $a$ takove ze $a_{n+1} = a_n$\\
$a = \frac{a^2}{4} + 1$ ; $a = 2$\\
z toho vime, ze pokud posloupnost konverguje, tak to bude k dvojce.\\
\\
Dale snadno vidime, ze $\forall n \in \mathbb{N}: a_n \leq a_{n+1}$ \dots posloupnost je neklesajici\\
\\
Dale snadno vidime ze $\forall a_n < 2; n \in \mathbb{N}: \frac{a^2_n}{4} + 1 < 2$ \dots "nepreskocime dvojku az se k ni budeme priblizovat"\\
\\
Jelikoz $a_1 < 2$, posloupnost je neklesajici a 2 nijak 
nepreskocime, pak\\
\doubleunderline{posloupnost konverguje k 2}\\


\end{document}