\documentclass[a4paper]{article}
\usepackage[utf8]{inputenc}
\usepackage{amsmath}
\usepackage{amsfonts}
\usepackage{amssymb}
\usepackage{hyperref}
\setlength{\parindent}{0in}
\usepackage{fancyhdr}
\usepackage[
    left   = 1.0 in,
    right  = 1.0 in,
    top    = 1.5 in,
    bottom = 1.5 in,
]{geometry}

\usepackage[czech]{babel} % recommended if you write in Czech

\begin{document}

\pagestyle{fancy}
\rhead{David Nápravník}

\section*{1. HW}
\subsection*{1}
\subsubsection*{a}
Mejme turinguv stroj $M$ = (Q, $\sum$, $\delta$, $q_0$, F) takovy ze:
\begin{itemize}
    \item mnozina stavu Q
    \item abeceda $\sum$=\{0, 1, $\lambda$\}
    \item prechodova funkce $\delta$
    \item pocatecni stav $q_0$
    \item mnozina koncovych stavu F
\end{itemize}

Binarni reprezentaci volime little-endian (nejmene vyznamny bit je vlevo).\\
Pouzijeme jednostranou nekonecnou pasku, jejiz krajni symbol je $\lambda$ a vpravo bude za cislem nekonecne nul.\\
Po dokonceni programu ze slusnosti vratime hlavu na zacatek.

Pro pricteni jednicky pouzijeme nasledujici princip:
\begin{enumerate}
    \item zacneme ve stavu increase s hlavou ukazujici na prvni bit cisla
    \item pokud vidime 1 zapiseme 0 a jdeme doprava, ponechame stav increase
    \begin{enumerate}
        \item tento krok opakujeme, dokud nezpropagujeme jednicku
    \end{enumerate}
    \item pokud vidime 0 zapiseme 1 a prejedeme hlavou na zacatek pasky, HOTOVO
\end{enumerate}


\subsubsection*{b}
\begin{itemize}
    \item $q_0$ = increase
    \item F = Done
    \item $\delta$:
    \begin{itemize}
        \item $\delta$(increase, 0) = (goBack, 1, L)
        \item $\delta$(increase, 1) = (increase, 0, R)
        \item $\delta$(goBack, 0) = (goBack, 0, L)
        \item $\delta$(goBack, 1) = (goBack, 1, L)
        \item $\delta$(goBack, $\lambda$) = (DONE, $\lambda$, R)
    \end{itemize}
\end{itemize}


\subsection*{2}
Mejme turinguv stroj $M$ = (Q, $\sum$, $\delta$, $q_0$, F) takovy ze:
\begin{itemize}
	\item abeceda $\sum$=\{a..z\}
	\item instrukce Z=\{L, R\}
	\item stavy Q=\{$\alpha ... \delta$\}
\end{itemize}

Pak $M'$ = (Q', $\sum$, $\delta$', $q_0$, F') bude turinguv stroj takovy ze:
\begin{itemize}
	\item abeceda $\sum$ zustava stejna
	\item instrukce dostanou moznost nedelat nic: Z'=\{L, R, N\}
	\item stavy ze prenasobenim instrukcemi ztrojnasobi na Q'=\{$\alpha$, $\alpha L$, $\alpha R$ ... $\delta$, $\delta L$, $\delta R$\}
	\item prechodova $\delta$' funkce se zmeni z:
	\begin{itemize}
        \item $\delta$(q, c) = (q', c', Z)
		\item na:
        \item $\delta$(q, c) = (qZ, c', N)
        \item $\delta$(qZ, c') = (q', \ , Z)
	\end{itemize}
\end{itemize}

\end{document}